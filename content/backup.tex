\section{Backup}
\begin{frame}
	\frametitle{Why does $\gamma_\mu\Psi^\mu$ transform like a dirac spinor.}
	The Dirac part of the vector-spinor spinor transform like
	\begin{equation*}
		\Psi^{\mu'} = \Lambda^\mu_{\ \nu}M(\Lambda) \Psi^\nu
	\end{equation*}
	Also we need the property
	\begin{equation*}
		M(\Lambda)\gamma_\mu M^{-1}(\Lambda) = \Lambda^{\nu}_{\ \mu} \gamma_{\nu}
	\end{equation*}
	To show 
	\begin{align*}
		(\gamma_\mu\Psi^\mu)' = \gamma_\mu  \Lambda^\mu_{\ \nu}M(\Lambda) \Psi^\nu
		= M(\Lambda)\gamma_\mu M^{-1}(\Lambda)M(\Lambda) \Psi^\nu
		= M(\Lambda) \gamma_\nu\Psi^\nu
	\end{align*}
\end{frame}
\begin{frame}
	\frametitle{Yet another formalism?}
	Very brief summary of the spinor-index formalism\footnote{Used in \fullcite[109]{landaulifschitzIV}}
	Rarita-Schwinger-Equation:
	\begin{align*}
		\hat p^{\delta \dot\gamma}\eta^{\dot \beta}_{\alpha\delta} &= m \zeta^{\dot \beta \dot \gamma}_{\alpha} \\
		\hat p_{\delta \dot\gamma}\zeta^{\dot \beta\dot \gamma}_{\alpha} &= m \eta^{\dot \beta}_{\alpha\delta} 
	\end{align*}
	Dottet and undottet indices with values $1, 2$ \enquote{labeling columns and	rows of the Pauli-Matrices}.
	Lowering/rising with 
	\begin{equation}
		g_{\alpha\beta}=
		\begin{pmatrix}
			0 & 1\\
			-1 & 0
		\end{pmatrix}
	\end{equation}
	This formalisim was the original one used by Fiertz and Pauli to describe spin $\sfrac{3}{2}$ and spin $2$ particles.
	\vspace{1em}
\end{frame}
