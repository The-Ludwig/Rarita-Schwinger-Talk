\section{Where to find Rarita-Schwinger Fermions}
\begin{frame}
	\frametitle{Delta resonances}
	Should be described by Rarita-Schwinger fields in effective theories.
	\begin{itemize}
		\item Inconsistencies are either ignored\footnote{\fullcite{Pascalutsa}} and pertubation theory is carried on 
		\item Or fixed with \enquote{ad-hoc} solutions 
		\item describe with the $(\sfrac{3}{2}, 0)\oplus(0, \sfrac{3}{2})$ representation, which does not suffer from superluminal travel, but also requires 
			some \enquote{ad-hoc} fixes \footnote{\fullcite{pure}}
	\end{itemize}	
\end{frame}
\begin{frame}
	\frametitle{Supergravity}
	Quick summary from someone who is new to supersymmetry/supergravity\footnotemark[1]:
	\footnotetext[1]{Here is the way down the rabbithole: \fullcite{weinberg2005quantum}}\\
	Each known boson is in a multiplet with a partner fermion and vice versa. Their spin can only differ by half an integer. 
	\pause	
	\begin{itemize}
		\item Gauge-Boson of gravity is the spin-2 graviton
		\item Its superpartner is the spin-$\sfrac{3}{2}$ gravitino
		\item It is massless in unbroken theory, gains mass similar to the Higgs-Mechanism by eating up the goldstino
		\item No Velo-Zwanziger problems 
	\end{itemize}	
\end{frame}
\begin{frame}
	
	\centering
	\alert{Thanks for listening! Any Questions?}
	
\end{frame}
