\section{The Lorentz Group}
\begin{frame}{The Lorentz Transformations}
	\begin{align*}
		x^\mu =
		\begin{pmatrix}
			t \\
			\vec x
		\end{pmatrix}
		\qquad
		x_{\mu} =
		g_{\mu\nu} x^{\nu}
		=
		\begin{pmatrix}
			t \\
			-\vec x
		\end{pmatrix}
		\qquad
		g_{\mu\nu}
		= \symup{diag}(1, -1, -1, -1)
		\qquad
		\symup{c} = 1
	\end{align*}
	Lorentz-Transformations are \emph{all} transformations $x_\mu\rightarrow x'_\mu$, which leave the spacetime distance
	\begin{align}
		\label{eqn:s2}
		s^2 = \l(x_2-x_1\r)_\mu\l(x_2-x_2\r)^\mu=(t_2-t_1)^2-(\vec x_2 - \vec x_1)^2
	\end{align}
	invariant.
	It is easy to see, that these transformations must be affine tranformations:
	\begin{align}
		\label{eqn:trans}
		x'^\mu = L(\Lambda, a; x)^\mu = \Lambda^\mu_{\ \nu}x^\nu + a^\mu
	\end{align}
	\eqref{eqn:s2} into \eqref{eqn:trans}:
	\begin{align*}
		a^\mu \in \symbb{R} \qquad g_{\alpha \beta} = g_{\mu\nu} \Lambda^\mu_{\ \alpha}\Lambda^\nu_{\ \beta}
	\end{align*}
\end{frame}
\begin{frame}{The Group Axioms}
	Lorentz Transformations fulfill the group axioms $\rightarrow$ Poincaré  or inhomogeneous Lorentz Group (LG)

	\centering
	\begin{tabular}{l l l}
		1. & $L(\Lambda, a)\circ L(\Lambda', a') \in \mathcal L$ & $L\l(\Lambda, a; L(\Lambda', a'; x)\r) = L(\Lambda\Lambda', \Lambda a'+a; x)$                                                      \\
		2. & Associativity:                                      & $\l(L(\Lambda, a)\circ L(\Lambda', a')\r)\circ L(\Lambda'', a'')= L(\Lambda, a)\circ \l(L(\Lambda', a')\circ L(\Lambda'', a'')\r)$ \\
		3. & Identity:                                           & $L(\symbb{1}_4, \vec 0; x) = x$                                                                                                    \\
		4. & Inverse:                                            & $L(\Lambda, a)^{-1}=L(\Lambda^{-1}, -\Lambda^{-1}a)$
	\end{tabular}

\end{frame}
\begin{frame}
	\frametitle{The Group Structure}

	\begin{tabular}{l l l}
		Inhomogeneous LG             & $L(\Lambda, a)$                                                                             & Unconnected          \\
		$\rightarrow$Space Inversion & $L(P, 0) = L(\symup{diag}(1, -1, -1, -1), 0)$                                               &                      \\
		$\rightarrow$Time Reversal   & $L(T, 0) = L(\symup{diag}(-1, 1, 1, 1), 0)$                                                 &                      \\
		Homogeneous LG               & $L(\Lambda, 0)$                                                                             & Unconnected subgroup \\
		Translations                 & $L(0, a)$                                                                                   & Connected subgroup   \\
		Rotations                    & $L(\symbf R_4, 0) =  L\l(\begin{pmatrix}
					1      & \vec 0^{\symup{T}}     \\
					\vec 0 & \symbf{R(\vec \alpha)}
		\end{pmatrix} , 0\r)\quad \symbf{R} \in \symup{SO(3)}$ & \vspace{1em}Connected subgroup   \\
		Boosts                       & $L(\symbf B, 0) = L\l(
			\begin{pmatrix}
					\gamma         & -\gamma \vec v^{\symup{T}}                                      \\
					-\gamma \vec v & \symbb{1} + \vec v \vec v^{\symup{T}} \frac{\gamma-1}{\vec v^2}
				\end{pmatrix}
		, 0\r)$                      & Connected subgroup                                                                                                 \\
		Restricted LG                & $L(\symbf R_4, 0) \cup L(\symbf B, 0)$                                                      & Connected subgroup
	\end{tabular}
\end{frame}
\begin{frame}
	\frametitle{The Lie-Algebra of the restricted LG}
	Transformations close to unity:
	\begin{align*}
		\Lambda^{\mu}_{\ \nu} = \delta^{\mu}_{\ \nu} + \omega^{\mu}_{\ \nu}
	\end{align*}
	$\omega^{\mu}_{\ \nu}$ is infinitesimal.
	\begin{align*}
		\Rightarrow g_{\alpha \beta} = g_{\mu\nu} \l(\delta^{\mu}_{\ \alpha} + \omega^{\mu}_{\ \alpha}\r)\l(\delta^{\nu}_{\ \beta} + \omega^{\nu}_{\ \beta}\r)
		= g_{\alpha\beta}+\omega_{\alpha\beta}+\omega_{\beta\alpha}+\symcal O (\omega^2) \Rightarrow  \omega_{\alpha\beta}= -\omega_{\beta\alpha}
	\end{align*}
	\pause
	Let $\symbf M(\Lambda)$ be a representation of the restricted LG: $\symbf M(\Lambda)\symbf M(\Lambda')=\symbf M(\Lambda\Lambda')$
	\begin{align*}
		\symbf M (1+\omega) = \symbf 1 + \frac{1}{2} i\omega_{\mu\nu} \symbf J^{\mu\nu} + \symcal O (\omega^2)
	\end{align*}
	Choose $\symbf J^{\mu\nu}=-\symbf J^{\nu\mu}$
	\begin{align*}
		\symup M(\Lambda)\symup M(1+\omega)\symup M(\Lambda)^{-1}
		= \symup M(\Lambda (1+\omega)\Lambda^{-1}) \\
		\dots \Rightarrow i[\symbf J^{\mu\nu}, \symbf J^{\rho\sigma}] = g^{\nu\rho}\symbf J^{\mu\sigma} - g^{\mu\rho} \symbf J^{\nu\sigma} - g^{\sigma\mu} \symbf J^{\rho\nu} + g^{\sigma\nu} \symbf J^{\rho\mu}
	\end{align*}
\end{frame}
\begin{frame}
	\frametitle{Representation of the restricted LG}
	Since $\symbf J^{\mu\nu}= -\symbf J^{\nu\mu}$, it has 6 independent components. We choose:
	\begin{align*}
		\symbf{\vec J} = (\symbf J^{23},\symbf J^{31},\symbf J^{12}) & \qquad \symbf{\vec K} = (\symbf J^{01},\symbf J^{02},\symbf J^{03}) \\
		\Rightarrow [\symbf J_i,\symbf J_j]                          & = i \epsilon_{ijk} \symbf J_k                                       \\
		\Rightarrow [\symbf J_i,\symbf K_j]                          & = i \epsilon_{ijk} \symbf K_k                                       \\
		\Rightarrow [\symbf K_i,\symbf K_j]                          & = -i \epsilon_{ijk} \symbf K_k
	\end{align*}
	\pause
	\begin{align*}
		\symbf{\vec A}^\pm              & = \frac{1}{2}\l(\symbf{\vec J} \pm i\symbf{\vec K}\r) \\
		[\symbf A^\pm_i,\symbf A^\pm_j] & = i \epsilon_{ijk} \symbf A^\pm_k                     \\
		[\symbf A^\pm_i,\symbf A^\mp_j] & = 0
	\end{align*}
\end{frame}
\begin{frame}
	\frametitle{$(A, B)$ representation of the restricted LG}
	The restricted Lorentz group can be written as  $L = \symup{SO}(3)_l \oplus \symup{SO}(3)_r$.
	\\We find matrices satisfying the $\symbf{\vec A}^\pm$ algebra with the standard spin matrices:
	\begin{align*}
		\l(\symbf{A_3}^+\r)_{aa'}                   & = a \delta_{aa'}                               \\
		\l(\symbf{A_1}^+\pm i\symbf{A_2}^+\r)_{aa'} & = \delta_{a',a\pm1} \sqrt{(A\mp a)(A\pm a +1)} \\
		a                                           & = -A, -A+1, \dots, +A
	\end{align*}
	Likewise for $\symbf{\vec A}^-$ only with $B$ and $b$ $\Rightarrow\ (A, B)$ representation of the restricted Lorentz group
	\pause
	Since $\symbf{\vec J}=\symbf{\vec A}^++\symbf{\vec A}^-$, Fields transforming under the $(A, B)$ representation have components with spin
	\begin{align*}
		j = A+B, A+B-1, \dots, \abs{A-B}
	\end{align*}
\end{frame}
\begin{frame}
	\frametitle{Examples of representations}
	\begin{tabular}{l l l}
		$(0, 0)$                                         & Scalar                        & Boson                              \\
		$(\sfrac{1}{2}, 0)\text{ and }(0, \sfrac{1}{2})$ & Lefthanded and RH Weyl Spinor & Massless fermion                   \\
		$(\sfrac{1}{2}, \sfrac{1}{2})$                   & Vector                        & Boson                              \\
		$\bigotimes_{i=1}^N(\sfrac{1}{2}, \sfrac{1}{2})$ & $N$-Rank Tensor               &                                    \\
		$(\sfrac{1}{2}, 0)\oplus(0, \sfrac{1}{2})$       & Dirac Bispinor                & Fermions                           \\
		$(\sfrac{1}{2}, 1)\oplus(1, \sfrac{1}{2})$       & Rarita Schwinger Field        & Gravitino                          \\
		$(2, 0)\oplus (0, 2)$                            & Spin 2 Spinor                 & Graviton / Rieman Curvature Tensor \\
	\end{tabular}
\end{frame}
