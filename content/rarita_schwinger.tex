\section{The Rarita-Schwinger Field}
\begin{frame}
	\frametitle{The Spinor of the Rarita-Schwinger Field}
	Idea: Combine spin 1 $A^\mu$ and the spin $\sfrac{1}{2}$ Dirac Spinor $\Psi$ into $\Psi^\mu$.
	It transforms according to the
	\begin{align*}
		(\sfrac{1}{2}, \sfrac{1}{2})\otimes\l[(\sfrac{1}{2}, 0)\oplus(0, \sfrac{1}{2})\r] =
		(\sfrac{1}{2}, 1)\oplus(\sfrac{1}{2}, 0)\oplus(1, \sfrac{1}{2})\oplus(0, \sfrac{1}{2})
	\end{align*}
	To eliminate the Dirac component, we require
	\begin{align*}
		\gamma_\mu \Psi^\mu = 0
	\end{align*}
	which transforms like a Dirac spinor. (Remember: $\gamma_\mu$ does not transform)
	\pause
	Rarita-Schwinger-Equation:
	\begin{align*}
		\l(i\slashed{\partial} - m\r)\Psi^\mu = 0
	\end{align*}
\end{frame}
\begin{frame}
	\frametitle{Properties of the Rarita-Schwinger equation}
	Using $\{\gamma^\mu, \gamma^\nu\}_+=2g^{\mu\nu}\symbb{1}_4$:
	\begin{equation*}
		\l .
		\begin{aligned}
			\gamma_\mu \Psi^\mu                       & = 0 \\
			\l(i\gamma^\nu\partial_\nu - m\r)\Psi^\mu & = 0
		\end{aligned}
		\r \rbrace
		\l(-i\gamma^\nu\partial_\nu\gamma_\mu+2i \partial_\nu g^\mu_\nu - m\gamma_\mu\r)\Psi^\mu
		\pause
		= 2i \partial_\mu \Psi^\mu
		= 0
		\Rightarrow
		\quad\partial_\mu \Psi^\mu = 0
	\end{equation*}
	\pause
	These equations impose each 4 conditions on the RS-Spinor giving
	\begin{equation*}
		4\times4-2\times4 = 16-8 = 8 \text{ degrees of freedom }=
		\l \lbrace
		\begin{aligned}
			\text{Particle: } s     & =-\sfrac{3}{2},-\sfrac{1}{2},\sfrac{1}{2},\sfrac{3}{2} \\
			\text{Antiparticle: } s & =-\sfrac{3}{2},-\sfrac{1}{2},\sfrac{1}{2},\sfrac{3}{2}
		\end{aligned}\r .
	\end{equation*}
\end{frame}
\begin{frame}
	\frametitle{Properties of the Rarita-Schwinger equation}
	Solutions in the Rest-Frame are\footnote{For a general one see: \fullcite{shi2003solution}}:
	\begin{align*}
		\Psi^\mu(\sigma)_{\pm} = \sum_{\sigma', \sigma''} CG(\sigma; \sigma',\sigma'') \epsilon^\mu(\sigma') u_\pm(\sigma'') e^{\mp i mt}\quad
		u_+(\sigma) =
		\begin{pmatrix}
			w(\sigma) \\
			0
		\end{pmatrix}\quad
		u_-(\sigma) =
		\begin{pmatrix}
			0 \\
			w(\sigma)
		\end{pmatrix}                                           \\
		w(\sigma=\sfrac{1}{2}) = \begin{pmatrix}
			1 \\
			0
		\end{pmatrix}\quad
		w(\sigma=-\sfrac{1}{2}) = \begin{pmatrix}
			0 \\
			1
		\end{pmatrix}\quad
		\epsilon^\mu(\sigma=0)=i(0, 0, 0, 1)\quad
		\epsilon^\mu(\sigma=\pm 1) = \mp \frac{i}{\sqrt 2} (0, 1, \pm i, 0) \\
	\end{align*}
	Nonvanishing Clebsch-Gordan coefficents are:
	\begin{align*}
		CG(+\sfrac{3}{2}; 1, \sfrac{1}{2}) = 1\qquad
		CG(-\sfrac{3}{2}; -1, -\sfrac{1}{2}) = 1\quad               \\
		CG(+\sfrac{1}{2}; 1, -\sfrac{1}{2}) = \sqrt\frac{1}{3}\quad
		CG(+\sfrac{1}{2}; 0, +\sfrac{1}{2}) = \sqrt\frac{2}{3}\quad \\
		CG(-\sfrac{1}{2}; -1, +\sfrac{1}{2}) = \sqrt\frac{1}{3}\quad
		CG(-\sfrac{1}{2}; 0, -\sfrac{1}{2}) = \sqrt\frac{2}{3}\quad
	\end{align*}
\end{frame}
\begin{frame}
	\frametitle{Lagrangian}
	\begin{equation*}
		\symcal{L} = \frac{1}{2}\overline\Psi_\nu\l(-\epsilon^{\nu\mu\kappa\lambda}\gamma_5\gamma_\mu\partial_\kappa+\frac{1}{2}m[\gamma^\nu, \gamma^\lambda]\r)\Psi_\lambda
		\text{\footnotemark[1]}
	\end{equation*}
	\footnotetext[1]{Full derivation:\fullcite[334]{weinberg2005quantum}}
	\pause
	Getting back the field equations:
	\begin{align*}
		\frac{\delta \symcal{L}}{\delta \overline\Psi_\nu}=0=
		\l(-\epsilon^{\nu\mu\kappa\lambda}\gamma_5\gamma_\mu\partial_\kappa+\frac{1}{2}m[\gamma^\nu, \gamma^\lambda]\r)\Psi_\lambda
		= T^\nu                                                                                                                                                                      \\
		\overset{\partial_\nu T^\nu}{\Rightarrow}
		[\slashed{\partial},\gamma^\lambda]\Psi_\lambda=0 \qquad
		\overset{\gamma_\nu T^\nu}{\Rightarrow}
		3m\gamma^\lambda \Psi_\lambda = \epsilon^{\nu\mu\kappa\lambda}\gamma_5\gamma_\nu\gamma_\mu\partial_\kappa\Psi_\lambda = i[\slashed \partial, \gamma^\lambda]\Psi_\lambda = 0 \\
		\Rightarrow \partial_{\lambda}\Psi^\lambda = \frac{1}{2}\l\{\slashed \partial, \gamma_\lambda \r\} \Psi^\lambda = \frac{1}{2}[\slashed \partial, \gamma_\lambda]\Psi^\lambda = 0
	\end{align*}
	Using
	\begin{equation*}
		\gamma^\mu\gamma^\nu\gamma^\rho = \eta^{\mu\nu}\gamma^\rho + \eta^{\nu\rho}\gamma^\mu - \eta^{\mu\rho}\gamma^\nu - i\epsilon^{\sigma\mu\nu\rho}\gamma_\sigma\gamma^5
	\end{equation*}
\end{frame}
\begin{frame}
	\frametitle{Lagrangian}
	\begin{align*}
	    0 =	\l(-\epsilon^{\nu\mu\kappa\lambda}\gamma_5\gamma_\mu\partial_\lambda+\frac{1}{2}m[\gamma^\nu, \gamma^\lambda]\r)\Psi_\lambda \\
		\gamma_\lambda\Psi^\lambda = 0 \qquad \partial_\lambda\Psi^\lambda = 0\\
	\end{align*}
	\pause 
	\begin{align*}
		-\frac{1}{2}\gamma^\lambda\gamma^\nu \Psi_\lambda 
		= -m\Psi^\nu 
		= - \epsilon^{\mu\nu\kappa\lambda}\gamma_5\gamma_\nu\partial_\kappa \Psi_\lambda
		= -i\l(\gamma^\nu\gamma^\kappa\gamma^\lambda-g^{\nu\kappa}\gamma^\lambda-g^{\kappa\lambda}\gamma^\nu+g^{\nu\lambda}\gamma^\kappa\r)\partial_\kappa\Psi_{\lambda}
		= -i\slashed \partial \Psi^\nu\\
	\end{align*}
	\pause
	\centering
	\alert{
	With this lagrangian we arrive at the correct field equations:
	\begin{align*}
		\Rightarrow (i\slashed \partial-m)\Psi^\nu = 0
	\end{align*}}
\end{frame}
\begin{frame}
	\frametitle{But wait! Why don't we use $(\sfrac{3}{2}, 0)\oplus(0, \sfrac{3}{1})$?}
	\begin{itemize}
	\item Weinberg did the calculations in 1964\footnote{\fullcite{weinberg1964feynman}} to get an 8-Spinor.
	\item \enquote{Intricate} calculations\footnotemark[1]
		\item Requires a generalised set of $\gamma$-matrices
	\end{itemize}
	\footnotetext[1]{Detailed discussion of ATS: \fullcite{pure}}
	\pause 
	The antisymmetric Tensor-Spinor (ATS) is sometimes used
	\begin{equation*}
		\l[\l(1, 0\r)\oplus\l(0,1\r)\r]\otimes\l[\l(\sfrac{1}{2}, 0\r)\oplus\l(0,\sfrac{1}{2}\r)\r]
		=\l[\l(\sfrac{1}{2}, 0\r)\oplus\l(0,\sfrac{1}{2}\r)\r]\oplus\l[\l(\sfrac{3}{2}, 0\r)\oplus\l(0,\sfrac{3}{2}\r)\r]
		\oplus\l[\l(\sfrac{1}{2}, 1\r)\oplus\l(1,\sfrac{1}{2}\r)\r]
	\end{equation*}
	$\Psi^{\mu\nu}=-\Psi{\nu\mu}$. The unphysical components removed with $\gamma_\mu\Psi^{\mu\nu}=0$.
	\begin{itemize}
		\item DOF$=6\times 4-4\times 4=8$
		\item Propagator has also (more on that later) problems in discribing $\Delta$ resonances 
		\item Detailed discussion: \footnotemark[1]
	\end{itemize}
	\vspace{1em}	
\end{frame}
